% This file is part of Harish's Resume.
%
% Copyright (C) 2019  Harish Rajagopal <harish.rajagopals@gmail.com>
%
% Harish's Resume is free software: you can redistribute it and/or modify
% it under the terms of the GNU General Public License as published by
% the Free Software Foundation, either version 3 of the License, or
% (at your option) any later version.
%
% Harish's Resume is distributed in the hope that it will be useful,
% but WITHOUT ANY WARRANTY; without even the implied warranty of
% MERCHANTABILITY or FITNESS FOR A PARTICULAR PURPOSE.  See the
% GNU General Public License for more details.
%
% You should have received a copy of the GNU General Public License
% along with Harish's Resume.  If not, see <https://www.gnu.org/licenses/>.

% NOTE:
% "project" should be a custom environment to be used for a single project, internship, etc.
% "emptyproject" should be a custom command to be used for an empty listing for a single project, internship, etc.
% The specifications of both of the above should be as follows:
% First argument is an OPTIONAL link
% Second argument is the name of the project
% Third argument is the date/duration
% Fourth argument is the name of the mentor(s)

\section*{Projects}
\begin{itemize}

\item \begin{project}[https://github.com/rharish101/multistage-step-size-scheduling-minimax]{Multistage Step Size Scheduling for Minimax}{Mar '22 -- Sep '22}{\iffull{Prof.\ Dr.\ Niao He, }ETH Zürich}%chktex 8
    \item Master's thesis on analysing and experimenting with multistage schedulers for the learning rate on minimax problems in machine learning and deep learning.
    \iffull{
        \item Analysed the theoretical convergence rates for the \href{https://pytorch.org/docs/stable/generated/torch.optim.lr\_scheduler.StepLR.html\#torch.optim.lr\_scheduler.StepLR}{Step Decay} scheduler, with experimental validation.
        \item \textbf{Improved} upon traditional learning rate schedulers for image generation using Generative Adversarial Networks (GANs).%
    }
\end{project}

\item \begin{project}[https://github.com/rharish101/label-noise-and-generalization]{Label Noise, Schedulers and Generalization}{Sep '21 -- Jan '22}{\iffull{Prof.\ Dr.\ Thomas Hofmann, }ETH Zürich}%chktex 8
    \item Semester project to investigate various step-size schedules towards generalization in the presence of label noise.
    \iffull{
        \item \textbf{Discovered} that empirically, higher batch sizes can overfit on noisy examples.
        \item Showed that commonly-held views regarding batch sizes in noisy datasets do not hold with learning rate schedulers.
    }
\end{project}

\item \begin{project}[https://github.com/rharish101/DeFMO/]{Deblurring Fast Moving Objects with Learned Losses}{Feb '21 -- Jun '21}{\iffull{Prof.\ Dr.\ Marc Pollefeys, }ETH Zürich}%chktex 8
    \item Semester project to extend \href{https://arxiv.org/abs/2012.00595}{DeFMO}, a deep learning model that removes motion blur of Fast Moving Objects\iffull{ (FMOs)}.
    \iffull{
        \item Improved the existing DeFMO renderer using Generative Adversarial Networks.
        \item \textbf{Optimized and refactored} the codebase to be more performant, consume less memory, and be more maintainable.
    }
\end{project}

\item \begin{project}[https://arxiv.org/abs/2004.09803]{CovidAID:\ COVID-19 Detection Using Chest X-Rays}{Apr '20 -- Aug '20}{\iftoggle{onepage}{IIT Kanpur \& IIT Delhi \& AIIMS}{Prof.\ Vinay Namboodiri, Prof.\ Chetan Arora, Prof.\ Krithika Rangarajan}}%chktex 8
    \item Multi-university collaboration to develop a neural network for detecting COVID-19 with an extremely imbalanced dataset.
    \iffull{
        \item Improved significantly upon the results of \href{https://github.com/JordanMicahBennett/SMART-CT-SCAN_BASED-COVID19_VIRUS_DETECTOR}{Covid-Net} with \textbf{90.5\% accuracy and 100\% recall} for COVID-19.
        \item Researched \textbf{multi-modal} models that can leverage patient metadata to improve diagnosis.
    }
\end{project}

\item \begin{project}{Interactive Training of GANs}{Jan '19 -- \iffull{Apr '19, Aug '19 -- }Dec '19}{\iffull{Prof.\ Vinay Namboodiri, Prof.\ Chetan Arora, }IIT Kanpur}%chktex 8
    \item Semester project to research human-in-the-loop methods that improve \href{https://arxiv.org/abs/1406.2661}{Generative Adversarial Networks}\ifsingle{ (GANs)}.
    \iffull{
        \item The user's interaction is used to obtain as well as to modify the latent space mapping as required.
        \item \textbf{Achieved efficient improvement} in generation of sketches and digits on standard datasets.
    }
\end{project}

\item \begin{project}[https://github.com/rharish101/UGP1/]{Multi-Agent GANs for Image Super-Resolution}{Aug '18 -- Dec '18}{\iffull{Prof.\ Vinay Namboodiri, }IIT Kanpur}%chktex 8
    \item Semester project to develop a multi-agent generalisation of GANs for image super-resolution.
    \iffull{
        \item Four generators get the four sections of the input, and their outputs are combined to get the final image.
        \item Each generator pairs with a discriminator, while a global discriminator acts on the final output.
    }
\end{project}

\item \emptyproject[https://github.com/rharish101/PClub-Project]{Depression Therapy Chatbot}{May '17 -- Jul '17}{Programming Club, IIT Kanpur}%chktex 8
\item \emptyproject[https://github.com/rharish101/ACA-Project]{Reinforcement Learning in Atari Games}{Jan '17 -- Jul '17}{Association of Computing Activities, IIT Kanpur}%chktex 8

\end{itemize}

% NOTE:
% "project" should be a custom environment to be used for a single project, internship, etc.
% "emptyproject" should be a custom command to be used for an empty listing for a single project, internship, etc.
% The specifications of both of the above should be as follows:
% First argument is an OPTIONAL link
% Second argument is the name of the project
% Third argument is the date/duration
% Fourth argument is the name of the mentor(s)

\section*{Projects}
\begin{itemize}

\item \begin{project}{Teaching GANs Interactively}{Jan '19 -- Aug '20}{Prof.\ Vinay Namboodiri, Prof.\ Chetan Arora}%chktex 8
    \item Researched effective \textit{interaction mechanisms} with a human-in-the-loop for transfer-learning and quality improvement of \href{https://arxiv.org/abs/1406.2661}{Generative Adversarial Networks} (GANs).
    \item Achieved efficient improvement in generation of sketches and digits on standard datasets.
    \item A human `teaches' a GAN to generate the desired outputs, in contrast to popular approaches of GANs aiding users.
    \item The user's interaction is used to obtain as well as to modify the latent space mapping as required.
\end{project}

\item \begin{project}[https://arxiv.org/abs/2004.09803]{CovidAID:\ COVID-19 Detection Using Chest X-Rays}{Apr '20 -- Aug '20}{Prof.\ Vinay Namboodiri, Prof.\ Chetan Arora, Prof.\ Krithika Rangarajan}%chktex 8
    \item Developed a \href{}{CheXNet}-based CNN model to classify chest X-rays with an Imbalance Ratio of 45.3:1 for COVID-19.
    \item Improved significantly upon the results of \href{https://github.com/JordanMicahBennett/SMART-CT-SCAN_BASED-COVID19_VIRUS_DETECTOR}{Covid-Net} with 90.5\% accuracy and 100\% recall for COVID-19.
    \item Researched \textit{multi-modal} models that can leverage patient metadata to improve diagnosis.
    \item Created forms for patients at AIIMS New Delhi that were used to gather X-rays and metadata for the dataset.
\end{project}

\item \begin{project}[https://github.com/rharish101/UGP1/]{Multi-Agent GANs for Image Super-Resolution}{Aug '18 -- Dec '18}{Prof.\ Vinay Namboodiri}%chktex 8
    \item A \textit{Multi-agent generalisation} of \href{https://arxiv.org/abs/1609.04802}{SRGAN} inspired by \href{http://openaccess.thecvf.com/content_cvpr_2018/papers_backup/Ghosh_Multi-Agent_Diverse_Generative_CVPR_2018_paper.pdf}{MADGANs} for image super-resolution in TensorFlow.
    \item Four generators (with shared lower layers) get the four corner sections of the input (with a slight overlap), and their outputs are joined (negating the overlap) to get the final high-resolution image.
    \item Each generator pairs with a discriminator, while a global discriminator acts on the final output.
\end{project}

\item \begin{project}[https://github.com/rharish101/CS771-Project]{Higher-Order Optimisation in Deep Learning}{Sept '18 -- Nov '18}{Prof.\ Piyush Rai, CS771A Course Project}%chktex 8
    \item A survey on the use of \textit{quasi-Newton methods} in deep learning as part of a course.
    \item Surveyed \href{http://www.cs.toronto.edu/~jmartens/docs/Deep_HessianFree.pdf}{Hessian-Free optimisation}, \href{https://arxiv.org/abs/1511.01169}{AdaQN}, and \href{https://arxiv.org/abs/1311.2115}{Sum of Functions Optimiser (SFO)}.
    \item Benchmarked Hessian-Free optimisation on an MLP against the Adam and SGD optimisers in TensorFlow.
\end{project}

\item \begin{project}[https://github.com/rharish101/eye-in-the-sky]{7\textsuperscript{th} Inter-IIT Tech Meet (Silver Medal)}{Dec '18}{IIT Kanpur Contingent}
    \item Won \textit{2\textsuperscript{nd} place} for ``The Eye in the Sky'': Semantic segmentation of satellite images using a dataset of only 14 images.
    \item Trained the \href{http://discovery.ucl.ac.uk/10032237/7/David_08270673.pdf}{P-Net} architecture and tuned hyper-parameters using \href{https://papers.nips.cc/paper/4443-algorithms-for-hyper-parameter-optimization}{\textit{Tree of Parzen Estimators}} with \href{https://hyperopt.github.io/hyperopt/}{Hyperopt}.
    \item Augmented the dataset by slicing each image into multiple images with random rotations.
\end{project}

\item \emptyproject[https://github.com/rharish101/CS335A]{Compiler for Golang in Python}{Jan '19 -- Apr '19}{Prof.\ Amey Karkare, CS335A Course Project}%chktex 8

\item \emptyproject[https://nfcl.pythonanywhere.com]{No-Frills Cab Locator -- Android App}{Sept '18 -- Nov '18}{Prof.\ Nisheeth Srivastava, CS252A Course Project}%chktex 8

\item \emptyproject[https://cse.iitk.ac.in/users/rharish/sixth-tech-meet]{6\textsuperscript{th} Inter-IIT Tech Meet}{Dec '17 -- Jan '18}{IIT Kanpur Contingent}%chktex 8

\item \emptyproject[https://github.com/rharish101/ACA-Project]{Reinforcement Learning in Atari Games}{Jan '17 -- July '17}{Association of Computing Activities, IIT Kanpur}%chktex 8

\item \emptyproject[https://github.com/rharish101/PClub-Project]{Depression Therapy Chatbot}{May '17 -- July '17}{Programming Club, IIT Kanpur}%chktex 8

\item \emptyproject[https://github.com/DEVANSH99/Image\_cptning2018]{Visual Attention in Image Captioning (Mentored)}{May '18 -- July '18}{Programming Club, IIT Kanpur}%chktex 8

\end{itemize}

% NOTE:
% "project" should be a custom environment to be used for a single project, internship, etc.
% "emptyproject" should be a custom command to be used for an empty listing for a single project, internship, etc.
% The specifications of both of the above should be as follows:
% First argument is an OPTIONAL link
% Second argument is the name of the project
% Third argument is the date/duration
% Fourth argument is the name of the mentor(s)

\section*{Projects}
\begin{itemize}
\ifsingle{\setlength{\itemsep}{0mm}}

\item \begin{project}[https://github.com/rharish101/DeFMO/]{Deblurring by Synthesizing Blur}{Feb '21 -- June '21}{Prof.\ Marc Pollefeys}%chktex 8
    \item Semester project to remove motion blur of Fast Moving Objects (FMOs) in images and videos.
    \iffull{\item A Convolutional Neural Network de-blurs individual frames as well as generates `sub-frames', as if captured by a high-speed camera (i.e.\ temporal super-resolution).}
    \item Builds upon \href{https://arxiv.org/abs/2012.00595}{DeFMO},\iffull{ a generative model} which re-constructs sharp contours and time-varying complex appearance of FMOs\iffull{ that move over 3D trajectories with 3D rotation}.
\end{project}

\item \begin{project}{Teaching GANs Interactively}{Jan '19 -- \iftoggle{onepage}{}{Apr '19, Aug '19 -- }Dec '19}{Prof.\ Vinay Namboodiri, Prof.\ Chetan Arora}%chktex 8
    \item Researched\iffull{ effective} \iftoggle{onepage}{\textbf{human-in-the-loop}}{interaction} mechanisms\iffull{ with a human-in-the-loop} for\iffull{ transfer-learning and} improving quality of \href{https://arxiv.org/abs/1406.2661}{Generative Adversarial Networks} (GANs).
    \item \textbf{Achieved efficient improvement} in generation of sketches and digits on standard datasets.
    \item The user's interaction is used to obtain as well as to modify the latent space mapping as required.
\end{project}

\item \begin{project}[https://arxiv.org/abs/2004.09803]{CovidAID:\ COVID-19 Detection Using\iffull{ Chest} X-Rays}{Apr '20 -- Aug '20}{Prof.\ Vinay Namboodiri, Prof.\ Chetan Arora\iffull{, Prof.\ Krithika Rangarajan}}%chktex 8
    \item Developed a \href{}{CheXNet}-based CNN model to classify chest X-rays with an Imbalance Ratio of 45.3:1 for COVID-19.
    \item Improved significantly upon the results of \href{https://github.com/JordanMicahBennett/SMART-CT-SCAN_BASED-COVID19_VIRUS_DETECTOR}{Covid-Net} with \textbf{90.5\% accuracy and 100\% recall} for COVID-19.
    \item Researched \textbf{multi-modal} models \iftoggle{onepage}{to use}{that can leverage} patient metadata\iffull{ to improve diagnosis}.
\end{project}

\iftoggle{onepage}
{
    \item \emptyproject[https://github.com/rharish101/UGP1/]{Multi-Agent GANs for Image Super-Resolution}{Aug '18 -- Dec '18}{Prof.\ Vinay Namboodiri}%chktex 8
    \item \emptyproject[https://github.com/rharish101/eye-in-the-sky]{7\textsuperscript{th} Inter-IIT Tech Meet (Silver Medal)}{Dec '18}{IIT Kanpur Contingent}
}
{
    \item \begin{project}[https://github.com/rharish101/UGP1/]{Multi-Agent GANs for Image Super-Resolution}{Aug '18 -- Dec '18}{Prof.\ Vinay Namboodiri}%chktex 8
        \item A \textbf{Multi-agent generalisation} of \href{https://arxiv.org/abs/1609.04802}{SRGAN} inspired by \href{http://openaccess.thecvf.com/content_cvpr_2018/papers_backup/Ghosh_Multi-Agent_Diverse_Generative_CVPR_2018_paper.pdf}{MADGANs} for image super-resolution in TensorFlow.
        \item Four generators (with shared lower layers) get the four corner sections of the input (with a slight overlap), and their outputs are joined (negating the overlap) to get the final high-resolution image.
        \item Each generator pairs with a discriminator, while a global discriminator acts on the final output.
    \end{project}

    \item \begin{project}[https://github.com/rharish101/eye-in-the-sky]{7\textsuperscript{th} Inter-IIT Tech Meet (Silver Medal)}{Dec '18}{IIT Kanpur Contingent}
        \item Won \textbf{2\textsuperscript{nd} place} for ``Eye in the Sky'': Semantic segmentation of satellite images using a dataset of only 14 images.
        \item Trained the \href{http://discovery.ucl.ac.uk/10032237/7/David_08270673.pdf}{P-Net} architecture and tuned hyper-parameters using \href{https://papers.nips.cc/paper/4443-algorithms-for-hyper-parameter-optimization}{\textbf{Tree of Parzen Estimators}} with \href{https://hyperopt.github.io/hyperopt/}{Hyperopt}.
        \item Augmented the dataset by slicing each image into multiple images with random rotations.
    \end{project}
}

\iffull{\item \emptyproject[https://cse.iitk.ac.in/users/rharish/sixth-tech-meet]{6\textsuperscript{th} Inter-IIT Tech Meet}{Dec '17 -- Jan '18}{IIT Kanpur Contingent}}%chktex 8
\item \emptyproject[https://github.com/rharish101/ACA-Project]{Reinforcement Learning in Atari Games}{Jan '17 -- July '17}{\iftoggle{onepage}{ACA}{Association of Computing Activities}, IIT Kanpur}%chktex 8
\item \emptyproject[https://github.com/rharish101/PClub-Project]{Depression Therapy Chatbot}{May '17 -- July '17}{Programming Club, IIT Kanpur}%chktex 8

\end{itemize}
